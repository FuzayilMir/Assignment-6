\documentclass[a4paper,12pt]{article}
\usepackage{graphicx}
\usepackage{amsmath}
\usepackage{tkz-euclide}
\usepackage{booktabs}
\title{Assignment-6\\ Latex Report}
\author{Fuzayil Bin Afzal Mir}
\date{21/01/2021}
\begin{document}
	\maketitle
	

 \newpage
 \begin{itemize}
	    \item \Large\textbf{Exercise 2.29}
	\end{itemize}
	\section{Construct PQRS where PQ = 4,QR = 6,
	RS = 5,PS = 5.5 and PR = 7.}\\
    	
\subsection{Solution} \\
It is given that,
 PQ = 4,QR = 6,RS = 5,PS = 5.5 and PR = 7.\\
 
 From the given values we conclude PQRS is an quadrilateral with diagonal PR.\\
 
  
  
  
\textbf{Figure of quadrilateral PQRS.}

\begin{tikzpicture}[scale=1]
    \coordinate[label=right:$S$] (S) at (5,0);
    \coordinate[label=left:$R$] (R) at (0,0);
    \coordinate[label=above:$Q$] (Q) at (0.35,6);
    \coordinate[label=above:$P$] (P) at (4.25,5.5);
    
    \draw (P)--node[above] {$\textrm{4}$}
    (Q)--node[left] {$\textrm{6}$}
    (R)--node[below] {$\textrm{5}$}
    (S)--node[right] {$\textrm{5.5}$}
    (P)--node[above] {$\textrm{7}$}(R)
    \end{tikzpicture}\\
    \textbf{\underline{Note:}} {Figure generated using latex.}\\$$$$\\

\subsection{Figure of quadrilateral PQRS}
\begin{figure}
    \centering
    \includegraphics[width=10cm]{EX-2.29.png}
    \caption{Generated using python}
    \label{fig:2}
\end{figure}

\\

  \textbf{Download the python code used for generating the figure from here:}
 
 \fbox{
  \begin{lstlisting}
  https://github.com/FuzayilMir/Assignment-6/blob/main/EX-2.29.py
 \end{lstlisting}
 
 
 

}



















\end{document}



